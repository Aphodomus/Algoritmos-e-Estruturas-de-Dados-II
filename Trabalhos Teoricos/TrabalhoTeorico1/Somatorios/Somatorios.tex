\documentclass[a4paper, 12pt, fleqn, leqno]{article}
\usepackage[top=3.5cm, bottom=2.5cm,
 left=3cm, right=3cm]{geometry}
\usepackage[utf8]{inputenc}
\usepackage{amsmath, amsfonts, amssymb}
\usepackage{graphicx}
\usepackage{indentfirst}
\usepackage{cite}

\begin{document}
\thispagestyle{empty}

\begin{figure}[h!b]
\centering \includegraphics[scale=2.5]{logopuc.jpg}
\end{figure}

\begin{center}
{\Large \bf Pontifícia Universidade Católica de Minas Gerais} \\
{\large \bf Trabalho Algoritmos e Estruturas de Dados II}

\vspace{3 cm}

{\Large \bf Resumo Somatório}

\vspace{4.5 cm}

{\Large \bf Daniel Vitor de Oliveira Santos}

\vspace{6.5cm}

{\large \bf Belo Horizonte \\ 2021}

\end{center}
	\section{Introdução}
		Com frequência as fórmulas matemáticas requerem a adição de muitas
variáveis. O somatório ou notação sigma é uma forma simples e 
conveniente de abreviação usada para fornecer uma expressão concisa
para a soma dos valores de uma variável, ou soma de parcelas. Um
somatório é um simples processo de adicionar. A notação matemática utiliza um símbolo para representar de forma compacta o somatório de vários termos similares: o símbolo de somatório $\sum$, uma forma ampliada da letra grega maiúscula sigma. Ele é definido como:
			\begin{flushleft}
				$$ \sum_{i = m}^{n} x_{i} = x_{m} + x_{m+1} + x_{m+2} + \ldots + x_{n - 1} + x_{n}$$
			\end{flushleft}
			
		Onde {\bf i} é o {\bf índice} da soma ; {\bf ${ x_{i}}$} é uma variável indexada que representa cada {\bf termo da soma}; {\bf m} é o {\bf índice inicial} (ou limite inferior), e {\bf n} é o {\bf índice final} (ou limite superior). O {\bf i = m} representa que o índice {\bf i} começa igual a {\bf m}. O índice {\bf i} é incrementado em uma unidade a cada termo, e terminando em {\bf i = n}
		
	\section{Propriedades básicas}
		As propriedades do somatório facilitam o desenvolvimento de expressões algébricas com a notação do somatório. O principal objetivo é trabalhar a expressão até ela se tornar uma soma simples ou uma soma quadrada.
		\begin{itemize}
			\item Somatório de uma constante K
			O somatório de uma constante é igual ao produto do numero de termos pela constante {\bf K}.
			\begin{flushleft}
				$$ \sum_{i = m}^{n} K = K + K + \ldots + K = K \cdot n$$
			\end{flushleft}
			\item Somatório do produto de uma constante por uma variável
			O Somatório do produto de uma constante por uma variável é igual ao produto da constante pelo somatório da variável.
			\begin{flushleft}
				$$ \sum_{i = m}^{n} K \cdot x_{i} = K \cdot x_{i} + K \cdot x_{i + 1} + \ldots + K \cdot x_{n} = K \cdot \sum_{i = m}^{n} x_{i}$$
			\end{flushleft}
			\item Somatório de uma soma ou subtração de variáveis
			O Somatório de uma soma ou subtração de variáveis é igual a soma ou subtração dos somatórios dessas variáveis, como:
			\begin{flushleft}
				$$ \sum_{i = m}^{n} (x_{i} \pm y_{i} \pm z_{i}) = \sum_{i = m}^{n} x_{i} \pm \sum_{i = m}^{n} y_{i} \pm \sum_{i = m}^{n} z_{i}$$
			\end{flushleft}
		\end{itemize}
		
	\section{Referências}
	
	PETERNELLI A., Luiz, (2003). "CAPÍTULO 1 - Conceitos Introdutórios". Disponível em: <http://www.dpi.ufv.br/~peternelli/inf162.www.16032004/materiais/CAPITULO1.pdf>. Acesso em: 17 fev 2021.\\
	
	Summation, In: WIKIPÉDIA. Disponível em: <https://en.wikipedia.org/wiki/Summation>. Acesso em: 17 fev 2021.
	
\end{document}